% Für Bindekorrektur als optionales Argument "BCORfaktormitmaßeinheit", dann
% sieht auch Option "twoside" vernünftig aus
% Näheres zu "scrartcl" bzw. "scrreprt" und "scrbook" siehe KOMA-Skript Doku
\documentclass[12pt,a4paper,titlepage,headinclude,bibtotoc]{scrartcl}


%---- Allgemeine Layout Einstellungen ------------------------------------------

% Für Kopf und Fußzeilen, siehe auch KOMA-Skript Doku
\usepackage[komastyle]{scrpage2}
\pagestyle{scrheadings}
\automark[section]{chapter}
\setheadsepline{0.5pt}[\color{black}]

%keine Einrückung
\parindent0pt

%Einstellungen für Figuren- und Tabellenbeschriftungen
\setkomafont{captionlabel}{\sffamily\bfseries}
\setcapindent{0em}

\usepackage{caption}
\usepackage{multirow}

%---- Weitere Pakete -----------------------------------------------------------
% Die Pakete sind alle in der TeX Live Distribution enthalten. Wichtige Adressen
% www.ctan.org, www.dante.de

% Sprachunterstützung
\usepackage[ngerman]{babel}

% Benutzung von Umlauten direkt im Text
% entweder "latin1" oder "utf8"
\usepackage[utf8]{inputenc}

% Pakete mit Mathesymbolen und zur Beseitigung von Schwächen der Mathe-Umgebung
\usepackage{latexsym,exscale,amssymb,amsmath}

% Weitere Symbole
\usepackage[nointegrals]{wasysym}
\usepackage{eurosym}

% Anderes Literaturverzeichnisformat
%\usepackage[square,sort&compress]{natbib}

% Für Farbe
\usepackage{color}

% Zur Graphikausgabe
%Beipiel: \includegraphics[width=\textwidth]{grafik.png}
\usepackage{graphicx}

% Text umfließt Graphiken und Tabellen
% Beispiel:
% \begin{wrapfigure}[Zeilenanzahl]{"l" oder "r"}{breite}
%   \centering
%   \includegraphics[width=...]{grafik}
%   \caption{Beschriftung} 
%   \label{fig:grafik}
% \end{wrapfigure}
\usepackage{wrapfig}

% Mehrere Abbildungen nebeneinander
% Beispiel:
% \begin{figure}[htb]
%   \centering
%   \subfigure[Beschriftung 1\label{fig:label1}]
%   {\includegraphics[width=0.49\textwidth]{grafik1}}
%   \hfill
%   \subfigure[Beschriftung 2\label{fig:label2}]
%   {\includegraphics[width=0.49\textwidth]{grafik2}}
%   \caption{Beschriftung allgemein}
%   \label{fig:label-gesamt}
% \end{figure}
\usepackage{subfigure}
\usepackage{adjustbox}

% Caption neben Abbildung
% Beispiel:
% \sidecaptionvpos{figure}{"c" oder "t" oder "b"}
% \begin{SCfigure}[rel. Breite (normalerweise = 1)][hbt]
%   \centering
%   \includegraphics[width=0.5\textwidth]{grafik.png}
%   \caption{Beschreibung}
%   \label{fig:}
% \end{SCfigure}
\usepackage{sidecap}

% Befehl für "Entspricht"-Zeichen
\newcommand{\corresponds}{\ensuremath{\mathrel{\widehat{=}}}}

%Für chemische Formeln (von www.dante.de)
%% Anpassung an LaTeX(2e) von Bernd Raichle
\makeatletter
\DeclareRobustCommand{\chemical}[1]{%
  {\(\m@th
   \edef\resetfontdimens{\noexpand\)%
       \fontdimen16\textfont2=\the\fontdimen16\textfont2
       \fontdimen17\textfont2=\the\fontdimen17\textfont2\relax}%
   \fontdimen16\textfont2=2.7pt \fontdimen17\textfont2=2.7pt
   \mathrm{#1}%
   \resetfontdimens}}
\makeatother

%Si Einheiten
\usepackage{siunitx}

%c++ Code einbinden
\usepackage{listings}
\lstset{numbers=left, numberstyle=\tiny, numbersep=5pt}

%Differential
\newcommand{\dif}{\ensuremath{\mathrm{d}}}

%Boxen,etc.
\usepackage{fancybox}
\usepackage{empheq}

%Fußnoten auf gleiche Seite
\interfootnotelinepenalty=1000

%Dateien aus Unterverzeichnissen
\usepackage{import}

%Bibliography \bibliography{literatur} und \cite{gerthsen}
%\usepackage{cite}
\usepackage{babelbib}
\selectbiblanguage{ngerman}

\begin{document}

\begin{titlepage}
\centering
\textsc{\Large Anfängerpraktikum der Fakultät für
  Physik,\\[1.5ex] Universität Göttingen}

\vspace*{4.2cm}

\rule{\textwidth}{1pt}\\[0.5cm]
{\huge \bfseries
  Versuch 23\\[1.5ex]
  Röntgenstrahlung}\\[0.5cm]
\rule{\textwidth}{1pt}

\vspace*{3.0cm}

\begin{Large}
\begin{tabular}{ll}
Praktikant:
 	&  Felix Kurtz\\
 	&  Michael Lohmann\\

E-Mail: 
	&  felix.kurtz@stud.uni-goettingen.de\\
	& m.lohmann@stud.uni-goettingen.de\\

 Betreuer: & Phillip Bastian\\
 Versuchsdatum: &  11.03.2015\\
\end{tabular}
\end{Large}

\vspace*{0.8cm}

\begin{Large}
\fbox{
  \begin{minipage}[t][2.5cm][t]{6cm} 
    Testat:
  \end{minipage}
}
\end{Large}

\end{titlepage}

\tableofcontents

\newpage

\section{Einleitung}
\label{sec:einleitung}
In diesem Versuch sollen mithilfe der Bragg-Reflexion die Eigenschaften von Röntgenstrahlung untersucht werden.
Dabei kann zum einen die \emph{Bremsstrahlung} als auch die für jedes Material \emph{charakteristische Strahlung} beobachtet werden.
Außerdem werden noch zwei Naturkonstanten der Atomphysik bestimmt: das \emph{Plancksche Wirkungsquantum} und die \emph{Rydberg-Frequenz}.

\section{Theorie}
\label{sec:theorie}
\subsection{Röntgenröhre und Bragg-Reflexion}
In einer Röntgenröhre (vgl. Abb.\ref{fig:Aufbau}) werden Elektronen durch einen Glühdraht emittiert und anschließend auf eine Anode beschleunigt.
Dort wechselwirken sie mit dem Anodenmaterial und es entsteht elektromagnetische Strahlung.
Diese ist bei den typischen Beschleunigungsspannungen $U_A$, die sich bei mehren Kilo-Volt bewegen, energiereicher als Licht.
Man nennt sie im deutschen Sprachraum nach ihrem Entdecker \textsc{Wilhelm Conrad Röntgen} \emph{Röntgenstrahlung}. \\
\begin{figure}[!h]
	\centering
	\includegraphics[scale=0.7]{Aufbau.png}
	\caption{Aufbau einer Röntgenröhre mit Detektor. \cite[Datum: 02.01.15]{LP23}}
	\label{fig:Aufbau}
\end{figure}

Um die Strahlung zu charakterisieren, trifft sie auf ein Kristallgitter, dessen Atomabstand im Bereich der Wellenlänge liegt.
Dort kommt es zur \emph{Bragg-Reflexion}, welche in Abb.\ref{fig:Bragg} dargestellt wird.
Die Strahlen werden an mehreren Kristallschichten reflektiert.
Dabei kommt es zu Gangunterschieden, welche ein ganzzahliges Vielfaches $n$ der Wellenlänge sein müssen, damit es zur konstruktiven Interferenz kommt und man am Detektor ein Signal erkennt.
So lässt sich also der Ablenkwinkel $2\theta$ in eine bestimmte Wellenlänge $\lambda$ umrechnen:
\begin{align}
	2d\sin\theta=n\lambda\,.
	\label{eq:Bragg}
\end{align}
Dies nennt man die \emph{Bragg-Bedingung}.

\begin{figure}[!h]
	\centering
	\includegraphics[scale=0.7]{Bragg.png}
	\caption{Bragg-Reflexion schematisch. \cite[Datum: 02.01.15]{LP23}}
	\label{fig:Bragg}
\end{figure}

\subsection{Geiger-Müller-Zählrohr}
Die Röntgenstrahlung wird durch ein Geiger-Müller-Zählrohr detektiert.
Dort ionisiert sie nämlich Atome und setzt somit Elektronen frei.
Diese lösen eine Kaskade an weiteren Elektronen aus und ein Strom kann gemessen werden.
Da das Zählrohr eine gewisse Zeit benötigt, bis es ein neues Ereignis detektieren kann, (\emph{Totzeit}), müssen die Impulsraten $N$ (Impulse/Sekunde) durch folgende Formel korrigiert werden:
\begin{align}
	N_\text{korrigiert}=\frac{N_\text{gemessen}}{1-\tau\cdot N_\text{gemessen}}\,.
	\label{eq:totzeit}
\end{align}
Dabei ist $\tau$ die Totzeit, die sich typischerweise im Bereich von $100 ~ \mu$s befindet.
Man berechnet nämlich die Wahrscheinlichkeit mit ein, dass ein Ereignis stattgefunden hat, dieses aber nicht detektiert werden konnte.

\subsection{Bremsstrahlung}
Das Elektron überträgt Energie an das Photon, wenn es mit dem Anodenmaterial wechselwirkt.
Läuft dieser Prozess ideal ab, gibt das Elektron seine gesamte Energie $E=e\cdot U_A$ an das Photon ab, dessen Energie $E=h\cdot \nu=\frac{hc}{\lambda}$ beträgt.
Setzt man also beide Energien gleich, kann man die minimal erreichbare Wellenlänge $\lambda_\text{gr}$ berechnen:
\begin{align}
	\lambda_\text{gr}=\frac{hc}{e \cdot U_A}\,.
	\label{eq:grenzLambda}
\end{align}
Sie hängt nur von der Beschleunigungsspannung $U_A$ ab, da die Elektronenladung $e$, die Vakuum-Lichtgeschwindigkeit $c$ sowie das Plancksche Wirkungsquantum $h$ Naturkonstanten sind.\\
Da aber die wenigsten Photonen die gesamte Elektronenenergie erhalten, ergibt sich eine typische, kontinuierliche Intensitätsverteilung, deren Maximum sich bei einer etwas größeren Wellenlänge befindet.
Man nennt dies \emph{Bremsstrahlung}.

\subsection{Charakteristische Röntgenstrahlung}
Neben der Bremsstrahlung misst man auch noch die für das Anodenmaterial \emph{charakteristische Strahlung}.
Dabei entspricht die Photonen-Energie genau einer Differenz von zwei Energieniveaus: $E_\text{ph}=h\nu=E_s-E_f$.
Ist das untere Niveau $E_f$ die K-Schale, spricht man von K-Linien.
Der Übergang von der L-Schale auf die K-Schale wird mit K$\alpha$ bezeichnet, der von M-Schale mit K$\beta$.
Diese Energiedifferenzen der K-Linien kann man mithilfe des \emph{Moseley-Gesetzes}
\begin{align}
	\nu_K=R_\nu (Z-\sigma)^2\left(\frac{1}{n_f^2}-\frac{1}{n_s^2}\right)
	\label{eq:moseley}
\end{align}
berechnen.
Dabei ist $\nu_K$ die Frequenz des Photons, $R_\nu$ die \emph{Rydberg-Frequenz}, $Z$ die Kernladungszahl des Atoms, die um die Abschirmkonstante $\sigma$ verringert wird, und $n_s$ bzw. $n_f$ kennzeichnen das obere/untere Niveau.
Für die K-Schale ist $n=1$, für die L-Schale gilt $n=2$, etc.
Analog kann man auch die Energien für die anderen Linien berechnen.
Dabei muss aber die Kernladungszahl um die für jede Schale typische Abschirmkonstante $\sigma$ verringert werden.
Für den K$\alpha$-Übergang ist $\sigma_K=1$.\\

Die Intensität $I_K$ der charakteristischen Strahlung hängt über
\begin{align}
	I_K \sim I_A\cdot(U_A-U_K)^{3/2}
	\label{eq:IntChara}
\end{align}
von dem Anodenstrom $I_A$ sowie der Anodenspannung $U_A$ und dem Ionisationspotential $U_K$ ab.

\subsection{Absorption von Röntgenstrahlung}
Trifft Röntgenstrahlung auf ein Material, wird sie abhängig vom Absorptionskoeffizient $\mu$ des Materials nach dem \emph{Lambert-beerschen Gesetz}
\begin{align}
	I(d)=I_0\cdot e^{-\mu d}
	\label{eq:absorption}
\end{align}
abgeschwächt.
Dabei ist $I_0$ die anfängliche Intensität und $I(d)$ die Intensität nach der Länge $d$.
Der Koeffizient $\mu$ hängt auch von der Wellenlänge $\lambda$ der Röntgenstrahlung ab, es gilt nämlich die Beziehung 
\begin{align}
	\frac{\mu}{\rho}\sim\lambda^3\,.
	\label{eq:abs_koeff(lambda)}
\end{align}
Dabei ist $\rho$ die Dichte des Materials und der Quotient $\frac{\mu}{\rho}$ der \emph{Massenschwächungskoeffizient}.


\section{Durchführung}
\label{sec:durchfuehrung}
Zuerst wird das Anodenmaterial notiert.
Da wir mit der Eisen-Röhre gearbeitet haben, werden die einzustellenden Werte nur für diese angegeben.\\
Nach Anschalten der Röntgenröhre, des Computers und Starten des Messprogramms \textsc{measure} wird zuerst die charakteristische Strahlung vermessen.
Im Programm stellt man dazu folgende Werte ein: Anodenspannung $U=25\,$kV, Anodenstrom $I=1\,$mA, Winkelschritt $\Delta\theta=0.1^\circ$ und Integrationszeit $\Delta t=2\,$s sowie den zu vermessenden Winkelbereich.
Dieser ist für jedes Anodenmaterial unterschiedlich, für Eisen soll zwischen $\theta_\text{min}=3^\circ$ und $\theta_\text{max}=80^\circ$ gemessen werden.\\
Danach vermisst man für die Spannungen $U=23, 26, 29, 32 \text{ und } 35\,$kV die Grenzwellenlänge und die Intensität der charakteristischen Strahlung, also für Eisen den Winkelbereich zwischen $3^\circ$ und $15^\circ$ sowie zwischen $17^\circ$ und $24^\circ$.
Wieder ist $I=1\,$mA, $\Delta\theta=0.1^\circ$ und $\Delta t=2\,$s.\\
Als nächstes werden die Absorptionskanten von Kupfer und Nickel vermessen.
In der Ablage auf dem Gerät befindet sich ein Kästchen mit Absorptionsfolien.
Man wählt die entsprechenden mit deiner Dicke von 0.025 mm.
Die weiteren Werte betragen $U=25\,$kV, $I=1\,$mA, $\Delta\theta=0.1^\circ$ und $\Delta t=30\,$s.\\
Als letztes wird in einem Bereich zwischen $8^\circ$ und $16^\circ$ mit $\Delta\theta=1^\circ$ die Absorption verschiedener Materialien (verschiedene Kernladungszahlen) vermessen.
Dazu wird als erstes ein Spektrum ohne Absorber aufgenommen, dann 0.08 mm dickes Aluminium sowie 0.025 mm dickes Zink und Zinn.
Die weiteren Werte betragen $U=25\,$kV, $I=1\,$mA und $\Delta t=30\,$s.
\newpage

\section{Auswertung}
\label{sec:auswertung}
Die verwendete Röntgenröhre hat eine Eisenanode und der Kristall besteht aus Lithiumfluorid mit einem Gitterabstand $d=201\,$pm.
Außerdem wird für die Totzeit des Detektors $\tau=100\,\mu$s angegeben.
So müssen zuerst alle Messwerte mit der Formel \eqref{eq:totzeit} korrigiert werden.
\subsection{Charakteristisches Spektrum von Eisen}
\subsubsection{Wellenlängen und Energien}
\begin{figure}[!htb]
	\centering
	\input{messung2.tex}
	\caption{Spektrum aus der Messung der charakteristischen Fe-Strahlung}
	\label{fig:messung2}
\end{figure}
In Abb.\ref{fig:messung2} ist die Zählrate gegen den Kristallwinkel aufgetragen.
Man sieht deutlich die Peaks der charakteristischen Strahlung von Eisen.
Der Winkel $\theta$ kann mit \eqref{eq:Bragg} in eine Wellenlänge $\lambda$ umgerechnet werden.
Der sich ergebende Fehler $\sigma_\lambda$ kann mit der Formel
\begin{align}
	\sigma_\lambda=\frac{2d}{n}\cos(\theta)\cdot\sigma_\theta
\end{align}
berechnet werden, welche der Fehlerfortpflanzung entstammt.
Die Wellenlänge kann nun in eine Energie umgerechnet werden:
\begin{align}
	E&=\frac{hc}{\lambda}\\
	\sigma_E&=\frac{hc}{\lambda^2}\cdot\sigma_\lambda
\end{align}
Dabei werden für die beiden Naturkonstanten folgende Werte verwendet: $h=6.626\cdot 10^{-34}\,$Js und $c=2.998\cdot 10^8\,$m/s.
Die sich so ergebenen Wert sind in Tabelle \ref{tab:FeChar} dargestellt.

\begin{table}[!htb]
	\centering
	\begin{tabular}{|c|c|c|c|c|c|}
		\hline
		&&&&\multicolumn{2}{c|}{Energie $E$ [eV]} \\		
		&$n$& Winkel $\theta$ & Wellenlänge $\lambda$ [pm] & Messwert & Lit. Wert\\
		\hline
		\multirow{2}*{K$_{\alpha}$} & 1 & $28.9^\circ \pm 0.2^\circ$ &  $194.3 \pm 1.3$& $6380 \pm 50$ & \multirow{2}*{6391, 6404}  \\
		& 2 & $74.1^\circ \pm 0.2^\circ$ & $193.3 \pm 0.2$ & $6414 \pm 14$  & \\
		\hline
		\multirow{2}*{K$_\beta$} & 1 & $26.0^\circ \pm 0.2^\circ$ & $176.2 \pm 1.3$ & $7040 \pm 60$ & \multirow{2}*{7058} \\
		& 2 & $60.8^\circ \pm 0.2^\circ$ &  $175.5 \pm 0.4$& $7065 \pm 17$ &\\
		\hline
	\end{tabular}
	\caption{Wellenlängen und Energien der charakteristischen Fe-Strahlung}
	\label{tab:FeChar}
\end{table}

\subsubsection{Abhängigkeit von der Anodenspannung}
In Abb.\ref{fig:messung3b} ist die Zählrate für verschiedene Spannungen im Bereich der Charakteristischen Strahlung gegen den Winkel aufgetragen.
Daraus kann man die Abhängigkeit aus \eqref{eq:IntChara} bestätigen.
Trägt man nämlich die Impulsrate der beiden Peaks gegen $(U_A-U_K)^{1.5}$ auf, ergibt sich in beiden Fällen eine Gerade (vgl Abb.\ref{fig:anode}).
Für die Rate wurde ein prozentualer Fehler von $1\%$ angenommen.
\begin{figure}[!htb]
	\centering
	\input{messung3b.tex}
	\caption{Messwerte im Bereich der charakteristischen Strahlung}
	\label{fig:messung3b}
\end{figure}

\begin{figure}[!htb]
	\centering
	\input{anode.tex}
	\caption{Charakteristische Strahlung: Intensität in Abhängigkeit der Anodenspannnung}
	\label{fig:anode}
\end{figure}

\subsection{Grenzwellenlänge der Bremsstrahlung und Plancksche Konstante}
\begin{figure}[!htb]
	\centering
	\input{messung3a.tex}
	\caption{Messwerte im Bereich der Grenzwellenlänge}
	\label{fig:messung3a}
\end{figure}

In Abb.\ref{fig:messung3a} sind für verschiedene Beschleunigungsspannungen die Impulsraten gegen den Kristallwinkel aufgetragen.
Man kann gut erkennen, dass bei höherer Spannung  der Grenzwinkel und damit auch die Grenzwellenlänge geringer wird.
Bei der Bestimmung des Grenzwinkels wird eine Unsicherheit von $0.2^\circ$ angenommen, da das Ablesen teilweise etwas schwierig war.
Über Formel \eqref{eq:Bragg} kann man wieder aus dem Winkel die zugehörige Wellenlänge bestimmen ($n=1$).
Trägt man nun das Produkt aus Anodenspannung $U_A$ und Grenzwellenlänge $\lambda_\text{gr}$ mit dem sich aus der Fehlerfortpflanzung folgenden Fehler $\sigma=U_A\cdot\sigma_{\lambda_\text{gr}}$ gegen die Anodenspannung auf, sollte sich nach Umstellen von \eqref{eq:grenzLambda} die Konstante $\frac{hc}{e}$ ergeben (vgl. Abb.\ref{fig:grenzLambda}).
Mit den Werten $e=1.602\cdot 10^{-19}\,$C und $c=2.998\cdot 10^8\,$m/s kann man nun das Planksche Wirkungsquantum bestimmen:
\begin{empheq}[box=\shadowbox]{align}
	h=(6.57 \pm 0.06)~10^{-34}\si{\joule\second}\,.
\end{empheq}

\begin{figure}[!htb]
	\centering
	\input{grenzLambda.tex}
	\caption{Produkt aus Beschleunigungsspannung und zugehöriger Grenzwellenlänge in Abhängigkeit der Spannung}
	\label{fig:grenzLambda}
\end{figure}

\subsection{Absorptionskanten und Rydberg-Konstante}
Nun befinden sich Kupfer- und Nickel-Folien im Strahlengang.
Trägt man nun die Impulsraten logarithmisch gegen den Winkel $\theta$ auf, erkennt man sehr gut die Absorptionskanten der beiden Materialien (vgl. Abb.\ref{fig:messung4}).
Aus der Winkelposition kann wieder mit \eqref{eq:Bragg} die zugehörige Wellenlänge berechnet werden.
Nutzt man $\nu=\frac{c}{\lambda}$ und stellt dann Formel \eqref{eq:moseley} um, kann man die Rydbergfrequenz und mit der Fehlerfortpflanzung auch den zugehörigen Fehler berechnen:
\begin{align}
	R_\nu&=\frac{c}{(Z-\sigma)^2\cdot\left(\frac{1}{n_f^2}-\frac{1}{n_s^2}\right)}\cdot \frac{1}{\lambda}\,,\\
	\sigma_{R_\nu}&=\frac{c}{(Z-\sigma)^2\cdot\left(\frac{1}{n_f^2}-\frac{1}{n_s^2}\right)}\cdot \frac{\sigma_\lambda}{\lambda^2}\,.
\end{align}
Geht man davon aus, dass ein K$\alpha$-Übergang für die Kante verantwortlich ist, muss man mit $\sigma=1$, $n_f=1$ und $n_s=2$ rechnen.
Ist der beobachtete Sprung jedoch ein K$\beta$-Übergang sind diese Werte\footnote{Quelle: http://de.wikipedia.org/wiki/Moseleysches_Gesetz} zu benutzen: $\sigma=1.8$, $n_f=1$ und $n_s=3$.
Die so berechneten Werte finden sich in Tabelle \ref{tab:Rydberg} wieder.
\begin{figure}[!htb]
	\centering
	\input{messung4.tex}
	\caption{Absorptionskanten von Kupfer und Nickel: Intensität logarithmisch gegen Winkel aufgetragen}
	\label{fig:messung4}
\end{figure}

\begin{table}[!htb]
	\centering
	\begin{tabular}{|c|c|c|c|c|}
		\hline
		& & & \multicolumn{2}{c|}{Rydbergfrequenz $R_\nu$ [$10^{15}~\si{Hz}$]} \\		
		& Winkel $\theta$ & Wellenlänge $\lambda$ [pm] & K$\alpha$-Übergang & K$\beta$-Übergang\\
		\hline
		Kupfer & $20.2^\circ \pm 0.2^\circ$ & $138.8 \pm 1.4$ & $3.67 \pm 0.04$ & $3.28 \pm 0.04$\\
		Nickel & $21.8^\circ \pm 0.2^\circ$ & $149.8 \pm 1.4$ & $3.67 \pm 0.04$ & $3.29 \pm 0.04$ \\
		\hline
	\end{tabular}
	\caption{Absorptionskanten von Kupfer und Nickel und die daraus berechnete Rydbergfrequenz}
	\label{tab:Rydberg}
\end{table}

\subsection{Absorptionskoeffizienten verschiedener Metalle}
\begin{figure}[!htb]
	\centering
	\input{messung5.tex}
	\caption{Absorption verschiedener Materialien: Impulsrate gegen Kristallwinkel}
	\label{fig:messung5}
\end{figure}
In Abb.\ref{fig:messung5} ist wieder die Zählrate gegen den Winkel aufgetragen.
Jedoch befanden sich verschiedene Materialien im Strahlengang, deren Absorption im Plot sichtbar wird.
So ist die Kurve mit der höchsten Intensität die ohne Absorber, die mit der niedrigsten ist mit Zink im Strahlengang aufgenommen worden.
Aus \eqref{eq:absorption} kann man durch Umstellen den Absorptionskoeffizienten berechnen:
\begin{align}
	\mu=\frac{1}{d}\cdot\ln\frac{I_0}{I(d)}\,.
\end{align}
Den so erhaltenen Wert muss man nun durch die Dichte de Materials teilen, um den Massenschwächungskoeffizienten zu berechnen.
Dabei werden folgende Werte für die Dichte verwendet: $\rho_\text{Al}=2.7\,$g/cm$^3$, $\rho_\text{Zn}=7.14\,$g/cm$^3$ und $\rho_\text{Sn}=7.28\,$g/cm$^3$.
Zur Überprüfung der Beziehung \eqref{eq:abs_koeff(lambda)} trägt man den Massenschwächungskoeffizient gegen die dritte Potenz der Wellenlänge auf (vgl. Abb.\ref{fig:abs_koeff(lambda)}).
Man erkennt bei allen drei Materialien den zu erwartenden linearen Zusammenhang.
\begin{figure}[!htb]
	\centering
	\input{messung6.tex}
	\caption{Massenschwächungskoeffizient in Abhängigkeit von der dritten Potenz der Wellenlänge}
	\label{fig:abs_koeff(lambda)}
\end{figure}

\section{Diskussion}
\label{sec:diskussion}
Bei diesem Versuch können die Praktikanten die Messergebnisse kaum negativ beeinflussen, da alles computergesteuert abläuft und justiert ist.

\subsection{Charakteristische Fe-Strahlung}
Die charkteristischen Linien von Eisen wurden mit einer Abweichung von weniger als $1\%$ gemessen (vgl Tab.\ref{tab:FeChar}).
Zudem liegen fast alle tatsächlichen Werte im Fehlerintervall.
Die Fehler wurden also richtig abgeschätzt.
Die zu erwartende K$\alpha$-Doppellinie konnte jedoch nicht aufgelöst werden.
Dazu hätte man eine kleinere Schrittweite $\Delta\theta$ des Winkels benötigt, da der Energieunterschied der beiden Niveaus sehr gering ist.\\
Die erwartete Abhängigkeit der Intensität von der Anodenspannung $I_K\sim (U_A-U_K)^{1.5}$ konnte auch bestätigt werden (vgl. Abb.\ref{fig:anode}).

\subsection{Bremsstrahlung und Planck-Konstante}
In Abbildung \ref{fig:grenzLambda} ist das Produkt aus Anodenspannung und Grenzwellenlänge wie  erwartet annähernd konstant.
Außerdem schneiden alle Fehlerintervalle die gefittete Konstante.
Das Experiment stimmt also mit der Theorie überein.\\
Der Messwert für die Planck-Konstante ist weniger als $1\%$ geringer als der tatsächliche Wert $h=6.626\cdot 10^{-34}\,$Js.
Außerdem liegt dieser Wert im Fehlerintervall.
Somit wurde das Plancksche Wirkungsquantum in diesem Versuch zufriedenstellend bestimmt.\\
Die hohen Impulsraten bei kleineren Winkeln als des Grenzwinkels (vgl. Abb.\ref{fig:messung3a} deuten jedoch auf einen Fehler im Aufbau, etc. hin, da sie theoretisch nicht möglich sind.
Warum sie trotzdem gemessen wurden, bleibt ein Rätsel.

\subsection{Rydberg-Konstante}
Der tatsächliche Wert der Rydbergfrequenz liegt bei $R_\nu=3.2899 \cdot 10^{15}\,\si\hertz$.
Somit müssen die beobachteten Absorptionskanten von Kupfer und Nickel von K$\beta$-Übergängen hervorgerufen werden, denn dann stimmt die gemessene Rydbergfrequenz mit dem Literaturwert überein (vgl. Tabelle \ref{tab:Rydberg}).
Die Abweichung ist sehr gering ($<1\%$) und außerdem liegt der Literaturwert in beiden Fehlerintervallen.

\subsection{Absorption}
In Abb.\ref{fig:abs_koeff(lambda)} kann man die erwartete Gesetzmäßigkeit $\frac{\mu}{\rho}\sim \lambda^3$ gut beobachten.
Die Messwerte liegen alle nahe der gefitteten Gerade.

\bibliography{literatur}
\bibliographystyle{babalpha}

\end{document}
