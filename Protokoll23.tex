% Für Bindekorrektur als optionales Argument "BCORfaktormitmaßeinheit", dann
% sieht auch Option "twoside" vernünftig aus
% Näheres zu "scrartcl" bzw. "scrreprt" und "scrbook" siehe KOMA-Skript Doku
\documentclass[12pt,a4paper,titlepage,headinclude,bibtotoc]{scrartcl}


%---- Allgemeine Layout Einstellungen ------------------------------------------

% Für Kopf und Fußzeilen, siehe auch KOMA-Skript Doku
\usepackage[komastyle]{scrpage2}
\pagestyle{scrheadings}
\automark[section]{chapter}
\setheadsepline{0.5pt}[\color{black}]

%keine Einrückung
\parindent0pt

%Einstellungen für Figuren- und Tabellenbeschriftungen
\setkomafont{captionlabel}{\sffamily\bfseries}
\setcapindent{0em}

\usepackage{caption}
\usepackage{multirow}

%---- Weitere Pakete -----------------------------------------------------------
% Die Pakete sind alle in der TeX Live Distribution enthalten. Wichtige Adressen
% www.ctan.org, www.dante.de

% Sprachunterstützung
\usepackage[ngerman]{babel}

% Benutzung von Umlauten direkt im Text
% entweder "latin1" oder "utf8"
\usepackage[utf8]{inputenc}

% Pakete mit Mathesymbolen und zur Beseitigung von Schwächen der Mathe-Umgebung
\usepackage{latexsym,exscale,amssymb,amsmath}

% Weitere Symbole
\usepackage[nointegrals]{wasysym}
\usepackage{eurosym}

% Anderes Literaturverzeichnisformat
%\usepackage[square,sort&compress]{natbib}

% Für Farbe
\usepackage{color}

% Zur Graphikausgabe
%Beipiel: \includegraphics[width=\textwidth]{grafik.png}
\usepackage{graphicx}

% Text umfließt Graphiken und Tabellen
% Beispiel:
% \begin{wrapfigure}[Zeilenanzahl]{"l" oder "r"}{breite}
%   \centering
%   \includegraphics[width=...]{grafik}
%   \caption{Beschriftung} 
%   \label{fig:grafik}
% \end{wrapfigure}
\usepackage{wrapfig}

% Mehrere Abbildungen nebeneinander
% Beispiel:
% \begin{figure}[htb]
%   \centering
%   \subfigure[Beschriftung 1\label{fig:label1}]
%   {\includegraphics[width=0.49\textwidth]{grafik1}}
%   \hfill
%   \subfigure[Beschriftung 2\label{fig:label2}]
%   {\includegraphics[width=0.49\textwidth]{grafik2}}
%   \caption{Beschriftung allgemein}
%   \label{fig:label-gesamt}
% \end{figure}
\usepackage{subfigure}
\usepackage{adjustbox}

% Caption neben Abbildung
% Beispiel:
% \sidecaptionvpos{figure}{"c" oder "t" oder "b"}
% \begin{SCfigure}[rel. Breite (normalerweise = 1)][hbt]
%   \centering
%   \includegraphics[width=0.5\textwidth]{grafik.png}
%   \caption{Beschreibung}
%   \label{fig:}
% \end{SCfigure}
\usepackage{sidecap}

% Befehl für "Entspricht"-Zeichen
\newcommand{\corresponds}{\ensuremath{\mathrel{\widehat{=}}}}

%Für chemische Formeln (von www.dante.de)
%% Anpassung an LaTeX(2e) von Bernd Raichle
\makeatletter
\DeclareRobustCommand{\chemical}[1]{%
  {\(\m@th
   \edef\resetfontdimens{\noexpand\)%
       \fontdimen16\textfont2=\the\fontdimen16\textfont2
       \fontdimen17\textfont2=\the\fontdimen17\textfont2\relax}%
   \fontdimen16\textfont2=2.7pt \fontdimen17\textfont2=2.7pt
   \mathrm{#1}%
   \resetfontdimens}}
\makeatother

%Si Einheiten
\usepackage{siunitx}

%c++ Code einbinden
\usepackage{listings}
\lstset{numbers=left, numberstyle=\tiny, numbersep=5pt}

%Differential
\newcommand{\dif}{\ensuremath{\mathrm{d}}}

%Boxen,etc.
\usepackage{fancybox}
\usepackage{empheq}

%Fußnoten auf gleiche Seite
\interfootnotelinepenalty=1000

%Dateien aus Unterverzeichnissen
\usepackage{import}

%Bibliography \bibliography{literatur} und \cite{gerthsen}
%\usepackage{cite}
\usepackage{babelbib}
\selectbiblanguage{ngerman}

\begin{document}

\begin{titlepage}
\centering
\textsc{\Large Anfängerpraktikum der Fakultät für
  Physik,\\[1.5ex] Universität Göttingen}

\vspace*{4.2cm}

\rule{\textwidth}{1pt}\\[0.5cm]
{\huge \bfseries
  Versuch 23\\[1.5ex]
  Röntgenstrahlung}\\[0.5cm]
\rule{\textwidth}{1pt}

\vspace*{3.0cm}

\begin{Large}
\begin{tabular}{ll}
Praktikant:
 	&  Felix Kurtz\\
 	&  Michael Lohmann\\

E-Mail: 
	&  felix.kurtz@stud.uni-goettingen.de\\
	& m.lohmann@stud.uni-goettingen.de\\

 Betreuer: & Phillip Bastian\\
 Versuchsdatum: &  11.03.2015\\
\end{tabular}
\end{Large}

\vspace*{0.8cm}

\begin{Large}
\fbox{
  \begin{minipage}[t][2.5cm][t]{6cm} 
    Testat:
  \end{minipage}
}
\end{Large}

\end{titlepage}

\tableofcontents

\newpage

\section{Einleitung}
\label{sec:einleitung}

\section{Theorie}
\label{sec:theorie}
\subsection{Röntgenröhre}
\begin{figure}[!h]
	\centering
	\includegraphics[scale=0.7]{Aufbau.png}
	\caption{Aufbau. \cite[Datum: 02.01.15]{LP23}}
\end{figure}
\begin{figure}[!h]
	\centering
	\includegraphics[scale=0.7]{Bragg.png}
	\caption{Bragg-Reflexion schematisch. \cite[Datum: 02.01.15]{LP23}}
\end{figure}
\begin{align}
	2d\sin\theta=n\lambda
\end{align}

\subsection{Geiger-Müller-Zählrohr}
\begin{align}
	N_\text{korrigiert}=\frac{N_\text{gemessen}}{1-\tau\cdot N_\text{gemessen}}
\end{align}

\subsection{charakteristische Röntgenstrahlung}
\begin{align}
	v_K=R_v (Z-1)^2\left(\frac{1}{n_f^2}-\frac{1}{n_s^2}\right)
\end{align}

\begin{align}
	v_L=R_v (Z-\sigma_L)^2\left(\frac{1}{n_f^2}-\frac{1}{n_s^2}\right)
\end{align}

\subsection{Abhängigkeit der Intensität von der Anodenspannung}
\begin{align}
	\lambda_\text{gr}=\frac{hc}{e \cdot U_A}
\end{align}

\begin{align}
	I_K \sim I_A\cdot(U_A-U_K)^{3/2}
\end{align}

\section{Durchführung}
\label{sec:durchfuehrung}

\section{Auswertung}
\label{sec:auswertung}
\subsection{Charakteristisches Spektrum von Eisen}
\subsubsection{Wellenlängen und Energien}
\begin{figure}[!htb]
	\centering
	\input{messung2.tex}
	\caption{Spektrum aus Messung 2}
\end{figure}
\begin{table}[!htb]
	\centering
	\begin{tabular}{|c|c|c|c|c|c|}
		\hline
		&&&&\multicolumn{2}{c|}{Energie $E$ [eV]} \\		
		&$n$& Winkel $\theta$ & Wellenlänge $\lambda$ [pm] & Messwert & Lit. Wert\\
		\hline
		\multirow{2}*{K$_{\alpha}$} & 1 & $28.9^\circ \pm 0.2^\circ$ &  $194.3 \pm 1.3$& $6380 \pm 50$ & \multirow{2}*{6391, 6404}  \\
		& 2 & $74.1^\circ \pm 0.2^\circ$ & $193.3 \pm 0.2$ & $6414 \pm 14$  & \\
		\hline
		\multirow{2}*{K$_\beta$} & 1 & $26.0^\circ \pm 0.2^\circ$ & $176.2 \pm 1.3$ & $7040 \pm 60$ & \multirow{2}*{7058} \\
		& 2 & $60.8^\circ \pm 0.2^\circ$ &  $175.5 \pm 0.4$& $7065 \pm 17$ &\\
		\hline
	\end{tabular}
\end{table}

\subsubsection{Abhängigkeit von der Anodenspannung}
\begin{figure}[!htb]
	\centering
	\input{messung3b.tex}
	\caption{Messwerte im Bereich der charakteristischen Strahlung}
\end{figure}

\begin{figure}
	\centering
	\input{anode.tex}
	\caption{Charakteristische Strahlung: Intensität in Abhängigkeit der Anodenspannnung}
\end{figure}

\subsubsection{Grenzwellenlänge der Bremsstrahlung und Plancksche Konstante}
\begin{figure}[!htb]
	\centering
	\input{messung3a.tex}
	\caption{Messwerte im Bereich der Grenzwellenlänge}
\end{figure}

\begin{figure}[!htb]
	\centering
	\input{grenzLambda.tex}
	\caption{Produkt aus Beschleunigungsspannung und zugehöriger Grenzwellenlänge in Abhängigkeit der Spannung}
\end{figure}

\begin{empheq}[box=\shadowbox]{align}
	h=(6.57 \pm 0.06)~10^{-34}\si{\joule\second}
\end{empheq}

\subsection{Absorptionskanten und Rydberg-Konstante}
\begin{figure}[!htb]
	\centering
	\input{messung4.tex}
	\caption{Absorptionskanten von Kupfer und Nickel: Intensität logarithmisch gegen Winkel aufgetragen}
\end{figure}
\begin{table}
	\centering
	\begin{tabular}{|c|c|c|c|}
		\hline
		& Winkel $\theta$ & Wellenlänge $\lambda$ [pm] & Rydbergfrequenz $R_\nu$\\
		\hline
		Kupfer & $20.2^\circ \pm 0.2^\circ$ & $138.8 \pm 1.4$ & $(3.67 \pm 0.04)\cdot 10^{15}~\si{Hz}$  \\
		Nickel & $21.8^\circ \pm 0.2^\circ$ & $149.8 \pm 1.4$ & $(3.67 \pm 0.04)\cdot 10^{15}~\si{Hz}$  \\
		\hline
	\end{tabular}
	\caption{Absorptionskanten von Kupfer und Nickel und die daraus berechnete Rydbergfrequenz}
\end{table}

\subsection{Absorptionskoeffizienten verschiedener Metalle}
\begin{figure}[!htb]
	\centering
	\input{messung5.tex}
	\caption{...}
\end{figure}

\begin{figure}[!htb]
	\centering
	\input{messung6.tex}
	\caption{...}
\end{figure}

\section{Diskussion}
\label{sec:diskussion}

\section{Anhang}

\bibliography{literatur}
\bibliographystyle{babalpha}

\end{document}
